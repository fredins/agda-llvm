% !TEX program = lualatex
\documentclass[10pt, twocolumn]{article}
% \usepackage{a4wide}

%\usepackage[a4paper, total={8.5in, 11in}]{geometry}
% \usepackage[a4paper]{geometry}
\usepackage[a4paper, margin=2.5cm]{geometry}
%\setlength{\textheight}{1.1\textheight}

% Support for Agda code.
\usepackage{agda}

\usepackage{fontspec}
\newfontfamily{\AgdaSerifFont}{Latin Modern Roman}
\newfontfamily{\AgdaSansSerifFont}{Latin Modern Sans}
\newfontfamily{\AgdaTypewriterFont}{Latin Modern Mono}
\renewcommand{\AgdaFontStyle}[1]{{\AgdaSansSerifFont{}#1}}
\renewcommand{\AgdaKeywordFontStyle}[1]{{\AgdaSansSerifFont{}#1}}
\renewcommand{\AgdaStringFontStyle}[1]{{\AgdaTypewriterFont{}#1}}
\renewcommand{\AgdaCommentFontStyle}[1]{{\AgdaTypewriterFont{}#1}}
\renewcommand{\AgdaBoundFontStyle}[1]{\textit{\AgdaSerifFont{}#1}}

% Small underscore
\renewcommand{\AgdaUnderscore}{\texttt{\_}}

% Bibliography
\usepackage[backend=biber, style=apa, natbib=true]{biblatex}
\addbibresource{../bibliography.bib}

% Packages
\usepackage{empheq} % for border around math
\usepackage{listings} % non-agda code blocks
\usepackage{amssymb} % math symbols
\usepackage[]{lmodern} % Latin modern font
\usepackage[labelfont=bf, textfont=it, justification=centering, singlelinecheck=false]{caption} 
\usepackage{caption}
\usepackage{subcaption}
\usepackage[skip=8pt plus1pt, indent=10pt]{parskip}

\usepackage[framemethod=tikz]{mdframed}
\mdfsetup{%
innertopmargin=1em,
innerbottommargin=1em,
innerleftmargin=1em,
innerrightmargin=1em,
linewidth=0.5pt,
}

\usepackage{color}
\newcommand{\hl}[2][lightgray]{\colorbox{#1}{#2}}

% Cosmetics
\linespread{1.25}

\usepackage{newunicodechar}
\usepackage{colonequals}
\newunicodechar{λ}{\ensuremath{\mathnormal{\lambda}}}
\newunicodechar{∀}{\ensuremath{\mathnormal{\forall}}}
\newunicodechar{∷}{\ensuremath{\mathnormal{\coloncolon}}}
\newunicodechar{ω}{\ensuremath{\mathnormal{\omega}}}
\newunicodechar{ℕ}{\ensuremath{\mathbb{N}}}

\newunicodechar{₁}{\ensuremath{{}_1}}
\newunicodechar{₂}{\ensuremath{{}_2}}
\newunicodechar{₃}{\ensuremath{{}_3}}
\newunicodechar{₄}{\ensuremath{{}_4}}
\newunicodechar{₅}{\ensuremath{{}_5}}
\newunicodechar{₆}{\ensuremath{{}_6}}
\newunicodechar{₇}{\ensuremath{{}_7}}
\newunicodechar{₈}{\ensuremath{{}_8}}
\newunicodechar{₉}{\ensuremath{{}_9}}


\usepackage{listings}
\lstdefinelanguage{Treeless}{
keywords={case, of, let, in}
}
\lstdefinelanguage{grin}{
keywords={case, of, store, update, unit, fetch}
}

\lstset{
frame=none,
framexleftmargin=0pt,
framesep=10pt,
stepnumber=1,
numbers=left,
numbersep=7pt,
numberstyle=\ttfamily\scriptsize\color[gray]{0.4},
basicstyle=\linespread{1}\ttfamily\fontseries{l}\selectfont,
keywordstyle=\linespread{1}\ttfamily\fontseries{b}\selectfont,
captionpos=b,
language=grin,
commentstyle=\it,
columns=flexible,
keepspaces=true,
mathescape=true,
escapechar=!,
}

\title{Precise reference counting for lazy functional languages with interprocedural points-to analysis}
\author{Martin Fredin \\ \texttt{fredinm@chalmers.se}}
\date{2023}

\begin{document}

\maketitle


\begin{code}[hide]
open import Agda.Builtin.Nat using (suc; zero; _+_) renaming (Nat to ℕ) 
open import Agda.Builtin.Strict using (primForce)

infixr 5 _∷_
data List A : Set where
  []  : List A
  _∷_ : (x : A) (xs : List A) → List A
\end{code}


% Motivation: 
% - Conflict between abstract code and runtime performance
%   • It is important for verified code to be performant: 
%       "Such security-critical code often lies on the critical path of larger subsystems; users therefore 
%        expect security-critical code to be not only secure and reliable, but also fast" \cite{ho2023}
%   • Many optimizations are hindered by module barriers, e.g. monomorphization, defunctionalization, and inlining. 
%   • As a result, polymorphic code are plagued with boxed types, dynamic dispatch and indirect function. 
%   • Non-strict evalutation semantics also encourage more allocations and indirect function calls. And purity prevents in-place mutation.
%   • Automatic memory management is often mandatory in high-level languages, which limits throughput and may introduce signficant delays.
%
% - In this paper, we address indirect function calls and automatic memory management.
%
% - We belive it is neccessary to utilse whole program compilation and interprocedural analysis to compile away abstractions. Hence GRIN.
%   TODO perhaps only focus on memory management and reference counting and mention that we chose GRIN address other concerns such as indirect 
%   function calls.
%
% - Precise reference counting is promising technique to enable in-place mutations for pure function languages (pure or immutable), 
%   and to avoid delays associtated with tracing-based garbage collectors. However, Perceus have only been applied for strict functional 
%   languages.
%
% Contribution: 
%   • We demonstrate the applicability of combining non-strict languages with precise reference counting.
%   • We implement a a compiler backend which compiles Agda to GRIN. Then, we perform transformations on the GRIN code to 
%     remove indirect function calls by an interprocedural analysis.  
%   • We implement a Perceus-like algorithm for the GRIN syntax which is much more low-level then previous implementations.
%   • We implement a GRIN interpreter and a LLVM code generator.
%
% Related Work and Discussions
% - ASAP uses compile-time garbage collection using interprocedural analysis. Would be interesting to apply some of these techniques to reduce reference counting operations.
% - Two objections two whole program compilation and reference counting are long compile times and innability to collect cyclic data structures, respectively.
%   We have been "avoiding" these uses for now. Hopefully, someone will come up with a nice solution. Some reference counting implementations like Swift and Python uses 
%   weak references. Koka do not handle cycles (Or doesn't have cycles). Roc does not have cycles. Lean does not have uncontrollable recursion nor mutable references, so it 
%   is impossible to create circular references.
% - Our precise reference counting algorithm targets GRIN but perhaps it can also be used in other simliar strict and low-level intermediate languages like the STG.
% - Our implementation is not ready for benchmarks. We would like to compile a larger subset of Agda, implement the rest of optimizing transformations, and reuse-analysis.
% - We would also like to formalize our precise reference counting algorithm.
%

% TODO
% - Focus more on motivation: problem of high-level verified code and efficient evaluation and memory management plays a huge role.
% - Maybe use non-strict/strict instead of lazy/eager 
% - Agda have normal order evaluation (This is maybe only true in compile-time normalization). Compiler writers can decide the most 
%   efficient evaluation order.
% - Remove result and create a separate section for code generator and interpreter.
% - Write in maybe abstract and introduction that we "cheat" by transforming non-strictness into a 
%   strict language 
% - Also write that we implement Perceus for a much more lower-level language (GRIN)
% - Need to make the section "Precise reference counting" more detailed
% - Fix bad writing
% - Think about which is the best de Brujin representation
% - Maybe make Perceus @fetch@ borrowing and then use dup/drop fusion

% - Section: introduction
%   • Write a motivation for the work. Why is this important for "outsiders"? 
%      - (Non-strictness is nice? Memory management / cache-fredliness is super important. Conflict between nice polymorphic code and performance)
%      - "Such security-critical code often lies on the critical path of larger subsystems; users therefore 
%         expect security-critical code to be not only secure and reliable, but also fast" \cite{ho2023}
% - Section: codegen/interpreter
%   • Mention that the layout is simple because we need to give it more thought
%     before implementing the reuse analysis
% - Section: Related work 
%

\begin{abstract}
Precise reference counting is a technique by \citeauthor{reinking2021} that uses ownership to deallocate objects as soon as possible. 
The algorithm is called Perceus, and as of this writing, it has only been implemented for eager functional languages.
This paper describes the implementation of a new lazy compiler back-end for the Agda programming language with 
precise reference counting. 
The compiler uses \citeauthor{boquist1999} and \citeauthor{johnsson1991}'s intermediate language GRIN to compile lazy programs. 
GRIN uses interprocedural points-to analysis to inline the evaluation of suspended computations.
We extend GRIN with a variant of Perceus, and demonstrate the applicability of combining lazy functional programming with precise reference counting by developing a GRIN interpreter and an LLVM code generator. 
% Our results suggest that strictness analysis is vital for mitigating space leaks.
% We conclude that...

\end{abstract}

\section{Introduction}

Reference counting \citep{collins1960} is a memory management technique which can detect and free resources as soon as they are no longer needed, allowing memory reuse and destructive updates. 
Common reference counting algorithms are easy to implement; each allocation contains an extra field which tracks the number of references to an object. 
When the reference count reaches zero, the heap space occupied by the object is reclaimed.
The number of references to a object is updated by interleaved reference counting operations (\texttt{dup} and \texttt{drop}), which increments and decrements the reference count at runtime. 
As a result of the interleaved collection strategy, memory usage remain low and the throughput is continuous throughout the computation\footnote{Reference counted programs may introduce pauses similar to tracing garbage collectors. For example, when decrementing a long linked list all at once.} \citep{jones1996}.
Still, tracing garbage collectors are usually favored over reference counting implementations due to cheaper allocations, higher throughput, and the ability to collect cyclic data structures.

\citet{reinking2021} reexamine reference counting with a new approach, utilizing static guarantees to make the algorithm precise so objects can be deallocated as soon as possible.
They present a formalized algorithm called Perceus, which ensures precision. 
Perceus is implemented in the functional language Koka, along with optimizations for reducing reference counting operations and reusing memory.
This, builds upon previous work by \citet{ullrich2021} in the Lean programming language and theorem prover. 

Both Koka and Lean are, however, eagerly evaluated. 
Lazy languages pose an extra challenge for compiler writers because of their unintuitive control flow. 
In this paper, we adapt the Perceus algorithm to a new lazy compiler back-end for the Agda programming language and proof assistant.
As a first step, we transform Agda into an intermediate language called GRIN \citep{johnsson1991}.
% FIXME "their unintuitive control" looks like it refererce to "compiler writers"

%\begingroup
%\large{TODO SKRIV RENT}
%\endgroup
%\normalfont
%\begin{itemize}
%\item expand on laziness (see boquist1996). Why is laziness important? Osasaki pure functional data structures.
%\item Lazy languages are hard optimize. For example, common implementations of laziness are implemented we indirect functional calls to 
%   evaluated suspended computations (johnsson1984, Peyton Jones 1992). Indirect function calls prevent optimization such as 
%   inlining and is proven to cache misses. 
%\item In this paper, we develop a new lazy compiler back-end for the Agda programming language and proof assistant.
%      Our implementation uses the intermediate language GRIN to compile lazy programs. GRIN uses an interprocedural points-to analysis
%      to eliminate all indirect function calls.
%\item We extend GRIN with variant of the Perceus algorithm for precise reference counting.
%\item Contributions:
%  \begin{itemize}
%  \item Demonstrate the applicability of combining lazy function programming with precise reference counting.
%  \item Adapt the Perceus algorithm to a much lower-level language than previous implementations
%  \item We develop GRIN interpreter and a LLVM code generator to compile GRIN to machine code.
%  \end{itemize}
%\end{itemize}

% \subsection{Syntax}
% TODO
% • We choosed GRIN because of powerful optimizations and simple. But, we belive that, precise reference counting could also be applied 
%   to languages like GHC's Core (PJ 96) and STG (92).

\section{Graph Reduction \\ Intermediate Notation}
In \citeyear{johnsson1991}, \citeauthor{johnsson1991} presented the Graph Reduction Intermediate Notation (GRIN) as an imperative version of the G-machine \citep{johnsson1984}, where lexically scoped variables are stored in registers instead of on the stack. 
Later, GRIN was reformulated with a more functional flavor \citep{boquist1995}.
In this project, we introduce an additional variant of GRIN adapted for the internal representation of Agda and precise reference counting. 
The syntax of our variant is shown in \mbox{Figure \ref{fig:grin-syntax}}.

A defining feature of GRIN is that suspended computations (thunks) and higher-order functions are defunctionalized \citep{reynolds1972}, by 
an interprocedural analysis called the heap points-to analysis.
As a result, all function calls through opaque pointers are eleminated. 
This also means that all GRIN values are in weak head normal form. 
However, in order to destinquish between complex values (thunks and closures) and simple values (literals and data types), 
future mentions of \emph{suspended computation} and \emph{weak head normal form} refers to the representation in the Agda program.


\begingroup
\setlength{\fboxsep}{1em} % padding for border
\begin{figure*}[htbp]
\centering
\begin{empheq}[box=\fbox]{align*}
% TODO fix rest of emph
&\begin{array}{l l l l}
term & ::=    & \emph{term} \; ; \; \lambda \emph{lpat} \rightarrow \emph{term}                        & \; \text{binding} \\
     & \; \mid  & \texttt{case} \; \emph{val} \; \texttt{of} \; \emph{term} \; \{\emph{calt}\,\}*  & \; \text{case} \\
     & \; \mid  & val \; \{val\}*                                           & \; \text{application} \\
     & \; \mid  & \texttt{unit} \; val                                      & \; \text{return value} \\
     & \; \mid  & \texttt{store} \; val                                     & \; \text{allocate new heap node} \\
     & \; \mid  & \texttt{fetch} \; \{tag\} \; n \;  \{i\}                  & \; \text{load heap node} \\
     & \; \mid  & \texttt{update} \; \{tag\} \; n \; \{i\} \; val           & \; \text{overwrite heap node} \\
     & \; \mid  & \texttt{unreachable}                                      & \; \text{unreachable} \\
\end{array} \\ \\
&\begin{array}{l l l l}
val & ::=     & \emph{tag} \; \{\emph{val}\,\}* & \; \text{constant node} \\
    & \; \mid & \emph{n} \; \{\emph{val}\,\}*   & \; \text{variable node} \\
    & \; \mid & \emph{tag}      & \; \text{single tag} \\
    & \; \mid & \emph{()}       & \; \text{empty} \\
    & \; \mid & \emph{lit}      & \; \text{literal} \\
    & \; \mid & \emph{n}        & \; \text{variable (de Bruijn index)} \\
    & \; \mid & \emph{def}      & \; \text{function definition} \\
    & \; \mid & \emph{prim}     & \; \text{primitive definition} \\
\end{array} \\ \\
&\begin{array}{l l l l}
lpat & ::=     & tag \; \{x\}* & \; \text{constant node pattern} \\
     & \; \mid & x \; \{x\}*   & \; \text{variable node pattern} \\
     & \; \mid & ()            & \; \text{empty pattern} \\
     & \; \mid & x             & \; \text{variable pattern} \\
\end{array} \\ \\
&\begin{array}{l l l l}
cpat & ::=     & tag \; \{x\}* & \; \text{constant node pattern} \\
     & \; \mid & tag             & \; \text{single tag pattern} \\
     & \; \mid & lit             & \; \text{literal pattern} \\
\end{array} \\ \\
&\begin{array}{l l}
\{...\}  & \text{means 0 or 1 times}    \\
\{...\}* & \text{means 0 or more times} \\
\end{array} 
\end{empheq}
\caption{GRIN syntax. }
\label{fig:grin-syntax}
\end{figure*}
\endgroup


\subsection{Code generation}
Currently, our compiler only accepts a subset of Agda which is lambda lifted, first-order, and monomorphic.
Following is Agda program which computes the sum of the first 100 numbers. 
%\begin{figure}[hb!]
\begin{code}[]
downFrom : ℕ → List ℕ
downFrom zero = []
downFrom (suc n) = n ∷ downFrom n 

sum : List ℕ → ℕ
sum [] = 0
sum (x ∷ xs) = x + sum xs

main = sum (downFrom 100) 
\end{code}
%\caption{Agda example program}
%\label{fig:agda-program}
%\end{figure}

Our back-end starts by converting the program into an untyped lambda calculus with let and case expressions called Treeless \citep{hausmann2015}.
The Treeless language uses administrative normal form, which means that the case scrutinee is always a variable, and the patterns cannot be nested or overlap.
Following is the Treeless representation of \AgdaFunction{downFrom}.

\begin{lstlisting}[language=Treeless, xleftmargin=24pt]
downFrom x7 = case x7 of
  0 → []
  _ → let x5 = 1
          x4 = _-_ x7 x5
          x3 = downFrom x4 in 
      _$∷$_ x4 x3
\end{lstlisting}
% cite De Bruijn 72
The implementation uses de Bruijn indices to represent variables, but this paper uses variable names to make it more readable. 
% uneccessary?
% During this phase, we also transform the program so applications only take variables as operands.

% TODO write that GRIN is low level comparatively.
% FIXME monadic operations instead of let expressions???
% TODO Be specific! Don't write "is simliar". Instead write "Translating Treeless to GRIN is straight forward..."
GRIN is similar to the Treeless syntax, but instead of let expressions we use the builtin state monad to bind variables and sequence operations.
The monadic operations are \lstinline{unit}, \lstinline{store}, \lstinline{fetch}, and \lstinline{update}. 
The bind operator ";" is infix and right-associative.
We can translate "\lstinline[language=Treeless]{let x = $\emph{val}$ in foo x}" lazily
by allocating the value and passing the pointer as an argument to the function: "\lstinline[]{store $\emph{val}$ ; λ x$_{ptr}$ → foo x$_{ptr}$}".
Here, \emph{val} must be a constant node value. 
A constant node is a tag followed by a sequence of arguments. 
We can pattern match on a tag with the case expression to determine the kind.
Tags are prefixed with either a "C" if it is a constructor value or an "F" for suspended function applications.
The node arguments are usually pointers to other heap-allocated nodes, but they can also be unboxed values.
For example, the boxed integer tag \lstinline{Cnat} accepts one unboxed integer.
Following is the corresponding GRIN representation of \AgdaFunction{downFrom}.

\begin{lstlisting}[xleftmargin=24pt]
downFrom x7 =
  eval x7 ; λ Cnat x6 →
  case x6 of
    0 → unit (C[])
    _ →
      store (Cnat 1) ; λ x5 →
      store (F_-_ x7 x5) ; λ x4 →
      store (FdownFrom x4) ; λ x3 →
      unit (C_$∷$_ x4 x3)
\end{lstlisting}

\subsection{Interpreter}

The semantics of our language are presented in \mbox{Figure \ref{fig:grin-semantics}}.
% TODO check so I didn't accidently plagiarise Boquist.
We use two semantic functions, $\mathcal{E}$ and $\mathcal{V}$, to assign semantic values to terms and syntactic values.
The semantics values are are a subset of syntactic values with an extra value representing undefined behaviour ($\bot$).
Additional, we have a new construct for heap allocated nodes. 
A heap allocated node consist of a reference count, a tag, and a sequence of values.

We have implemented the semantics as an interpreter, check that new transformations do not change the semantics.
Due to the lower level of GRIN, we can collect information about resource usage.
For instance, our example allocates the following nodes: 101 \lstinline{Cnat}, 101 \lstinline{FdownFrom}, 100 \lstinline{F_-_},  and 100 \lstinline{Fsum}.
And the worst-case bound for heap consumption is 202 nodes.
With this information we can estimate the performance characteristics of the program and identify optimization opportunities.

\begingroup
\setlength{\fboxsep}{1em} % padding for border
\begin{figure}[htbp]
\centering
\begin{empheq}[box=\fbox]{align*}
\begin{array}{l}
\mathcal{E} : \emph{Stack} \rightarrow \emph{Heap} \rightarrow \emph{Term} \rightarrow \emph{Value} \times \emph{Heap} \\
\mathcal{V} : \emph{Stack} \rightarrow Val \rightarrow \emph{Value} \times \emph{Heap} \\
\\
\emph{Value} = \emph{tag} \; \{\emph{val}\,\}* \mid \emph{tag} \mid \emph{loc} \mid \emph{()} \mid \emph{lit} \mid \bot \\
\emph{HeapNode} = \emph{ref-count} \; \emph{tag} \; \{\emph{val}\,\}* \\
\emph{Stack} = \emph{Var} \rightarrow  \emph{Loc} \\
\emph{Heap} = \emph{Loc} \rightarrow  \emph{HeapNode} \\
\end{array}
\end{empheq}
\caption{GRIN semantics. }
\label{fig:grin-semantics}
\end{figure}
\endgroup


\label{sec:analysis-and-transformations}
\subsection{Analysis and transformations}
The most important GRIN transformation is \emph{eval inlining}. 
\lstinline{eval} is a normal GRIN function which forces suspended computations to its weak head normal form.
There are no suspended computations in GRIN and all values are weak head normal form.
Thus, "suspended computation" and "weak head normal form" should always refer to the corresponding representation in the source program.
In the above function, "\lstinline{eval x7 ; λ Cnat x6 → ...}" evaluates the value at the pointer (\lstinline{x7}) to a boxed integer \lstinline{Cnat x6}.

Eval inlining generates a specialized \lstinline{eval} function for each call site.
To evaluate a suspended computation, we load the node from the heap using \lstinline{fetch}. 
Then, we pattern match on the possible nodes. 
Constructor nodes are already in weak head normal form so they are left unchanged.
Function nodes need to be evaluated by applying the arguments to the corresponding function.
Finally, the heap is updated with the evaluated value.

\begin{lstlisting}[xleftmargin=24pt]
downFrom x7 =
  (fetch x7 ; λ x34 →
   (case x34 of
      Cnat x36 → unit (Cnat x36)
      F_-_ x37 x38 → _-_ x37 x38
   ) ; λ x35 →
   update x7 x34 ; λ () →
   unit x35
  ) ; λ Cnat x6 →
  case x6 of
    0 → unit (C[])
    _ →
      store (Cnat 1) ; λ x5 →
      store (F_-_ x7 x5) ; λ x4 →
      store (FdownFrom x4) ; λ x3 →
      unit (C_$∷$_ x4 x3)
\end{lstlisting}

% TODO fix ugly
Eval inlining require a set of possible nodes for each abstract heap location.
The set needs to be relatively small, or otherwise an excessive amount of code will be generated.
We use the \emph{heap points-to} analysis \citep{johnsson1991}.
The analysis is interprocedural, meaning that multiple functions need to be analyzed together.
We will not go into detail about the algorithm, as it is thoroughly described in \citep{boquist1996}. 
Instead, this paper will only provide a general intuition of the algorithm.
Consider the inlined evaluation in \lstinline{downFrom}.
There are two tags in the case expression: \lstinline{F_-_} and \lstinline{Cnat}. 
\lstinline{F_-_} comes from the suspended recursive call inside \lstinline{downFrom}, and 
\lstinline{Cnat} is from the \lstinline{update} operation and the suspended call to \lstinline{downFrom} in the main function.

\begin{lstlisting}[xleftmargin=24pt]
main =
  store (Cnat 100) ; λ x20 →
  store (FdownFrom x20) ; λ x19 →
  sum x19 ; λ Cnat x18 →
  printf x18
\end{lstlisting}

\citeauthor{boquist1999}'s thesis presents 24 transformations divided into two groups: simplifying transformations and optimizing 
transformations \citep{boquist1999}. 
The simplifying transformations are necessary for the code generator and are all implemented, except \emph{inlining calls to apply}
which is used for partially applied functions. 
The optimizing transformations significantly alter the program and achieve similar effects to deforestation \citep{wadler1988} and listlessness \citep{wadler1984}.
% FIXME deforestation is the algorithm for getting listlessness/Treelessness
% TODO Instead, mention that it eliminates allocations and unboxing
We have only implemented \emph{copy propagation}.
This project aims to combine lazy functional programming with precise reference counting. 
Producing the most optimized code is out of the scope of this project. 
However, this is something we would like to explore in future research.

\section{Precise reference counting}
After the GRIN transformations, we insert reference counting operations to automatically manage memory.
We use an algorithm based on Perceus \citep{reinking2021}. 
Perceus is a deterministic syntax-directed algorithm for the linear resource calculus $λ₁$.
$λ₁$ is an untyped lambda calculus extended with explicit binds and pattern matching.
Our implementation uses GRIN which have explicit memory operations and different calling conventions.
In this section, we give brief a overview of the algorithm and discuss some challenges when adapting Perceus to GRIN. 
We also describe two optimizations: \emph{drop specialization} and \emph{dup/drop fusion}.

The algorithm uses two sets of resource tokens; an owned environment and a borrowed environment. 
Elements in the owned environment must be consumed exactly once. 
We consume a value by calling the function \lstinline{drop}, or by transfer ownership to another consumer.
An example of this is \lstinline{main} which require no reference counting operations because it transfers ownership of both of its allocations.
The allocation "\lstinline{store (Cnat 100) ; λ x20 →}" is consumed by the suspended computation "\lstinline{store (FdownFrom x20) ; λ x19 →}", which in turn is consumed by "\lstinline{sum x19}".
Elements in the borrowed environment can only be applied to non-consuming operations, such as pattern matching. 
We can promote an element from the borrowed environment to the owned environment by calling the function \lstinline{dup}.

\begin{figure*}[hp]
\begin{mdframed}
\centering
\setlength{\fboxsep}{0pt} % padding for border
\begin{subfigure}[t]{0.44\textwidth}
\centering
\begin{lstlisting}
sum x14 =
  fetch x14 [2] ; λ x40 →
  downFrom x40 ; λ x59 x60 x61 →
  case x59 of
    [] →
      update x14 (C[]) ; λ () →
      !\hl{drop x14 ; λ () →}!
      unit (Cnat 0)
    _$∷$_ →
      !\hl{dup x61 ; λ () →}!
      !\hl{dup x60 ; λ () →}!
      update x14 (C_$∷$_ x60 x61) ; λ () →
      !\hl{drop x14 ; λ () →}!
      store (Fsum x61) ; λ x10 →
      _+_ x10 x61
\end{lstlisting}
\caption{dup/drop insertion}
\label{fig:drop-insert}
\end{subfigure}\par\medskip
\begin{subfigure}[b]{0.44\textwidth}
\centering
\begin{lstlisting}
sum x14 =
  fetch x14 [2] ; λ x40 →
  downFrom x40 ; λ x59 x60 x61 →
  case x59 of
    [] →
      update x14 (C[]) ; λ () →
      fetch x14 [0] ; λ x499 →
      (case x499 of
         1 → free x14
         _ →
           PSub x499 1 ; λ x498 →
           update x14 [0] x498
      ) ; λ () →
      unit (Cnat 0)
    _$∷$_ →
      !\hl{dup x61 ; λ () →}!
      !\hl{dup x60 ; λ () →}!
      update x14 (C_$∷$_ x60 x61) ; λ () →
      fetch x14 [0] ; λ x501 →
      (case x501 of
         1 →
           !\hl{drop x61 ; λ () →}!
           !\hl{drop x60 ; λ () →}!
           free x14
         _ →
           PSub x501 1 ; λ x500 →
           update x14 [0] x500
      ) ; λ () →
      store (Fsum x61) ; λ x10 →
      _+_ x10 x61
\end{lstlisting}
\caption{drop specialization}
\label{fig:drop-spec}
\end{subfigure}
\begin{subfigure}[b]{0.44\textwidth}
\begin{lstlisting}[showlines=true, numbers=none]
sum x14 =
  fetch x14 [2] ; λ x40 →
  downFrom x40 ; λ x59 x60 x61 →
  case x59 of
    [] →
      update x14 (C[]) ; λ () →
      fetch x14 [0] ; λ x499 →
      (case x499 of
         1 → free x14
         _ →
           PSub x499 1 ; λ x498 →
           update x14 [0] x498
      ) ; λ () →
      unit (Cnat 0)
    _$∷$_ →
      update x14 (C_$∷$_ x60 x61) ; λ () →
      fetch x14 [0] ; λ x501 →
      (case x501 of
         1 →
           free x14
         _ →
           !\hl{dup x61 ; λ () →}!
           !\hl{dup x60 ; λ () →}!
           PSub x501 1 ; λ x500 →
           update x14 [0] x500
      ) ; λ () →
      store (Fsum x61) ; λ x10 →
      _+_ x10 x61


\end{lstlisting}
\caption{dup push-down and dup/drop fusion}
\label{fig:dup/drop-fusion}
\end{subfigure}
\end{mdframed}
\caption{Perceus transformations}
\label{fig:perceus}
\end{figure*}

Figure \ref{fig:drop-insert} presents the function \lstinline{sum} with reference counting operations inserted and highlighted in gray.
Many aspects of adapting Perceus to GRIN are present in \lstinline{sum}. 
For example, we can conclude that the operations \lstinline{fetch} and \lstinline{update} must be borrowing operations. 
This is evident because both \lstinline{fetch} and \lstinline{update} use \lstinline{x14} prior to it being explicitly dropped.
We can also conclude that \lstinline{fetch} bind variables that extend the owned environment.
Meanwhile, the bound variables of function applications extend the borrowed environment.

One of the goals of GRIN is to improve register utilization for lazy functional languages \citep{boquist1999}.
As such, the result of function applications, \lstinline{fetch}, and \lstinline{unit} are regular values stored in registers.
There is no point to reference count values in registers, so we need to treat pointer variables and non-pointer variables differently.
This is in contrast to the calculus which Perceus is defined for, $λ₁$, where all values except variables are heap-allocated \citep{reinking2021}.
Another difference is that functions in GRIN are not allowed to return unboxed pointers. 
Moreover, all function calls return explicit nodes rather than node variables. 
For example: "\lstinline{downFrom x40 ; λ x59 x60 x61 → ...}".
This i problematic because \lstinline{x60} and \lstinline{x61} are undefined when \lstinline{downFrom} returns the empty list.
Therefore, we are only allowed to \lstinline{dup} the variables once the tag is known.

As a result of GRIN's explicit memory operations, we can describe the Perceus primitives \lstinline{dup}, \lstinline{drop}, \lstinline{is-unique}, \lstinline{decref}, and \lstinline{free} with GRIN's existing constructs.
GRIN lacks a primitive for deallocating memory, so \lstinline{free} is just a foreign function call to libc's \lstinline{free}. 
However, our implementation of \lstinline{drop} is unsatisfactory. 
Nodes of different tags vary in arity and the arguments can be either boxed or unboxed.
Therefore, we need to pattern match on all possible tags to drop the child references. 
This is similar to the problem with a general \lstinline{eval} function (Section \ref{sec:analysis-and-transformations}). 
A crucial difference is that \lstinline{drop} is recursive, so we cannot always eliminate the general function by specialization and inlining.
In Figure \ref{fig:drop-spec}, we specialize both calls to \lstinline{drop}.
This eliminates all reference counting operations in the branch for the empty list.
Then, we push down the \lstinline{dup} operations and eliminate \lstinline{dup}/\lstinline{drop} pairs.
This optimization is called \emph{dup/drop fusion} and the result is presented in Figure \ref{fig:dup/drop-fusion}.
We can eliminate the general \lstinline{drop} function completely for our example program, but this is not always the case. 
\citeauthor{reinking2021} presents an additional optimization called \emph{reuse analysis}, which perfoms destructive updates when possible.
This transformation is not yet implemented.

\section{LLVM code generator}
We have implemented a GRIN interpreter and an LLVM code generator to check that our compiler works and that the programs reclaim all the allocated memory. 
Our implementation of the code generator is simple. 
GRIN does not demand a specific memory layout for the node values.
The only requirements are that tags should be unique and easy to extract \citep{boquist1999}.
We use an array of four 64-bit integer values for all heap-allocated nodes. 
The first two address spaces contain the reference count and the tag, respectively.
The node arguments occupy the rest of the array.
LLVM IR is a statically typed language, while GRIN is untyped.
This discrepancy is not an issue because all functions return nodes, and all variables are pointers or integers.
Hence, we store the pointers as integer values and convert them to the pointer type (\lstinline{ptr}) as required.
However, we will probably develop a type system for GRIN once we switch to a more sophisticated memory layout and implement the \emph{general unboxing} optimization \citep{boquist1999}.

\section{Result}

\begin{code}
{-# TERMINATING #-}
downFrom′ : ℕ → List ℕ
downFrom′ zero = []
downFrom′ (suc n) = 
  primForce n (λ n → n ∷ downFrom′ n)

sum′ : ℕ → List ℕ → ℕ
sum′ acc [] = acc
sum′ acc (x ∷ xs) = 
  sum′ (primForce x _+_ acc) xs

main′ = sum′ 0 (downFrom′ 100) 
\end{code}

This is common space leak problem amongs lazy functional programs \citep{wadler1987}



% \begin{itemize}
% \item Short text about tailcalls
% \item Currenlty we use libc's \texttt{malloc} and \texttt{free}, however, \citeauthor{pinto2023} suggest that these should be implemented in assembly.
% \end{itemize}

% There are 402 allocations in our example program (page 2): 

% \begin{itemize}
% \item Stack overflows (tail calls partly remedy this)
% \item Integer overflow
% \item The necessary parts of GRIN and Perceus is implemented but a lot of optimizations are 
      % left on the table. In GRIN we have mostly implemented the necessary simplifying transformations,
      % which turns GRIN into a state which is suitable for the code generator.
      % We haven't implemented drop specialization or heap reuse analysis.
% \end{itemize}

\section{Related work}

Reference counting implementations are uncommon in the litterature of lazy functional languages.
\citet{kaser1992} uses a reference counting for the lazy parallel language EQUALS.
They observe that reference counting minimizes memory contention\footnote{i.e. Multiple processes trying to access the same memory at once.} because memory operations are interleaved througout the computation.
Additionaly, they note that their reference counting implementation achieved greatly reduced heap usage and good memory locality.

To our knowledge, precise reference counting which utilizes ownership to free objects as soon as possible have not yet been implemented in any lazy functional language.
The Perceus algorithm have been implemented for multiple eager languages \citep{reinking2021, ullrich2021, teeuwissen2023, pinto2023}.


% 1. Reference counting in lazy languages. EQUAL.
% 2. Precise reference counting. Perceus. Lean, Koka, Roc.
%    FBIP. Based on Hofmann (2000).
% 3. GRIN. Interprocedural analysis.
% 4. Compile-time garbage collection. Linear types. ASAP.

% To our knowledge, Perceus is the main memory management technique in the Koka \citep{reinking2021}, Lean \citep{TODO}, and Roc \citep{teeuwissen2023}.
% There is also an experimental implementation in OCaml \citep{pinto2023}.
% All of these, are eager functional languages and mostly immutable.
% Our back-end uses lazy evaluation.


%\begin{itemize}
%\item GRIN \citep{podlovics2021}
%
%\item Our Perceus implementation is also unique because the it is adapted to much lower-level language with explicit memory operations and pointers.
%      Koka, Lean, and OCaml uses an intermediate languages simliar to the lambda calculus.
%      Roc is also implemented in low-level language.
%      But one major differnce is that Roc lacks pattern matching.
%      Roc have also a lot hard coded types like arrays.
%      Roc performs defunctionalization of higher-order functions simliar to how we use defunctionalization for on thunks. 
%      \citeauthor{teeuwissen2023} claims that this helps the reuse-analysis enables more instances of FBIP.
%
%\item We have not implemented the most powerful optimizations like \emph{Functional But In Place} (FBIP) and borrowing specialization.
%
%\item Cycles. Both Lean and Roc eager purely functional languages which makes it impossible to create cycles.
%      Cycles are possible in Koka since it allows mutable references.
%      The programmer is responsible for to break cycles manually by marking certain references a weak.
%      This strategy is also found in Swift and Python.
%      Agda is a total language so every program is productive regardless of evaluation order. 
%      Thus, in principle, we could turn to eager evaluation when there is a circular reference.
%      Furthermore, unspecified evaluation order enable many optimization.
%
%\item ASAP is an intresting language which uses compile-time garbage collection using interprocedural analysis.
%      Can we leverage some information for the points-to analysis and sharing analysis?
%\item Lobster is another intresting reference counted language with support for compile-time reference counting.
%
%\item Other Agda Backends. MAlonzo inserts a lot of unsafeCoerce (which kills optimzations) because Treeless is untyped citep{hausmann2015}.
%
%\end{itemize}


% Perceus
% • To our knowledge, Perceus is the main memory management technique in the following languages Koka, Lean, and Roc. 
% • There has also been an experimental implementation in OCaml.
% • All of these languages are eagerly evaluated.
% 
% Cycles
% • 
% • 
% • 
%   
%   



% Related Work and Discussions
% - ASAP uses compile-time garbage collection using interprocedural analysis. Would be interesting to apply some of these techniques to reduce reference counting operations.
% - Two objections two whole program compilation and reference counting are long compile times and innability to collect cyclic data structures, respectively.
%   We have been "avoiding" these uses for now. Hopefully, someone will come up with a nice solution. Some reference counting implementations like Swift and Python uses 
%   weak references. Koka do not handle cycles (Or doesn't have cycles). Roc does not have cycles. Lean does not have uncontrollable recursion nor mutable references, so it 
%   is impossible to create circular references.

% - Our precise reference counting algorithm targets GRIN but perhaps it can also be used in other simliar strict and low-level intermediate languages like the STG.
% - Our implementation is not ready for benchmarks. We would like to compile a larger subset of Agda, implement the rest of optimizing transformations, and reuse-analysis.
% - We would also like to formalize our precise reference counting algorithm.

% Lean and Koka: reference counting and reuse
% Swift: automatic reference counting (ARC), and plans of borrowing
% AST \& Rust: safe manual memory management through proofs and the borrow checker, respectively
\section{Conclusions and future work}
In this paper, we combine lazy functional programming with precise reference counting.
Our implementation compiles Agda to GRIN and extends GRIN with precise reference counting instructions. 
Then, we compile GRIN to LLVM and show that the program reclaims all the memory.
Currently, our implementation allocates a lot of nodes. 
In future research, we would like to minimize allocations by implementing the rest of the GRIN optimizing transformations and the Perceus reuse analysis.
We are also interested in developing a type system for GRIN and compiling a larger set of Agda programs.

% \begin{itemize} 
% \item implement reuse analysis and the rest of transformation to avoid uneccessary allocations and laziness.
%\item It would cool to benchmark our work but we lack many optimization.
%\item The current implementation of GRIN and Perceus have not yet implementation many optimization transformations. For example, GRIN lacks function inlining, generalized unboxing, and arity raising. Perceus lacks it two most import transformations; drop specialization and reuse analysis.
%\item Another huge optimization on is a strictness analysis. 
%      Agda is total language where all programs are productive. 
%      As such, the evalutation order of side-effect free code does not matter.
%      There is not risk in accidently making a productive program unproductive, 
%      which is an major part of strictness analysis (Mycroft 1980). Perhaps,
%      it would be better to use strictness by default and have a non-strict analysis.
%\item Add reuse and borrowing
%\item Utilse GRIN's whole program compilation strategy and the heap points-to analysis to statically determine unshared values during compile time, and thus minimizing the number of reference countinging operations.
%\item It would also be intresting to utilse 0-modality (erasure) in Agda's type system, and later also 1-modality when Agda gets it.
%\item In the current naive implementation drop enumerates all the possible tags 
%\end{itemize}

% Lambda lifting
% The rest of the GRIN transformations and optimizations
% Partial applications (P-tags and the apply operation)
% Different allocator? mimalloc?
% Using type information (multiplicites: 0, (1), ω)


\printbibliography

\end{document}


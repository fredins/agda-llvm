% !TEX program = lualatex
\documentclass{article}

% Support for Agda code.
\usepackage{agda}

\usepackage{fontspec}
\newfontfamily{\AgdaSerifFont}{Latin Modern Roman}
\newfontfamily{\AgdaSansSerifFont}{Latin Modern Sans}
\newfontfamily{\AgdaTypewriterFont}{Latin Modern Mono}
\renewcommand{\AgdaFontStyle}[1]{{\AgdaSansSerifFont{}#1}}
\renewcommand{\AgdaKeywordFontStyle}[1]{{\AgdaSansSerifFont{}#1}}
\renewcommand{\AgdaStringFontStyle}[1]{{\AgdaTypewriterFont{}#1}}
\renewcommand{\AgdaCommentFontStyle}[1]{{\AgdaTypewriterFont{}#1}}
\renewcommand{\AgdaBoundFontStyle}[1]{\textit{\AgdaSerifFont{}#1}}

% Bibliography
\usepackage[backend=biber, style=apa, natbib=true]{biblatex}
\addbibresource{../bibliography.bib}

% Packages
\usepackage{empheq} % for border around math
\usepackage{listings} % non-agda code blocks


% Cosmetics
\linespread{1.25}


\setlength{\fboxsep}{1em} % padding for border

\usepackage{newunicodechar}
\usepackage{colonequals}
\newunicodechar{λ}{\ensuremath{\mathnormal{\lambda}}}
\newunicodechar{∀}{\ensuremath{\mathnormal{\forall}}}
\newunicodechar{₁}{\ensuremath{{}_1}}
\newunicodechar{∷}{\ensuremath{\mathnormal{\coloncolon}}}
\newunicodechar{ω}{\ensuremath{\mathnormal{\omega}}}

\title{Optimizing lazy functional languages with precise reference counting}
\author{Martin Fredin}
\date{June 2023}

\begin{document}

\maketitle

\begin{code}[hide]



open import Data.List using (List; _∷_; [])

private 
  variable
    A B : Set

\end{code}

\section{Introduction}

All values in purely functional languages are immutable.
Immutability is important because it limits shared state, and makes it easier to reason about the program.
In practice, this means that a function such as \AgdaFunction{map} returns a new list instead of modifying the input.
\begin{code}
map : (A → B) → List A → List B
map f []       = []
map f (x ∷ xs) = f x ∷ map f xs
\end{code}
This is great if the input list (\AgdaDatatype{List}\AgdaSpace\AgdaGeneralizable{A}) is used later in the program. 
Otherwise, it is better to reuse the allocation for \AgdaDatatype{List}\AgdaSpace\AgdaGeneralizable{A} by updating the list in place. 
This avoids both the deallocation of \AgdaDatatype{List}\AgdaSpace\AgdaGeneralizable{A} and the allocation of \AgdaDatatype{List}\AgdaSpace\AgdaGeneralizable{B}.
To preserve the semantics of the program, destructive updates should only be perfomed on values which are garanteed to be non-shared. 
We call a value shared if there is multiple references to it.

Reference counting \citep{collins1960} is a memory management technique which can detect and free resources as soon as they are no longer needed, allowing memory reuse and destructive updates. 
Common reference counting algorithms are easy to implement; each allocation contains an extra field which tracks the number of references to a value, and reclaims the heap space once the reference count drops to zero. 
As a result of the interleaved collection strategy, memory usage remain low and the throughput is continuous throughout the computation\footnote{Not entirely true.} \citep{jones1996}.
However, ...

\begin{itemize}
\item Lower throughput than tracing GC
\item Not very good at handling short lived values which is most values (especially in pure FP)
\item \citet{reinking2021} and \citet{ullrich2021} minimize RC operations through precise reference counting and a linear calculus.
% \item Perceus are more adapt to handle short lived valus, or rather unique values, because the fast path 
%     in drop 
\item more...
\end{itemize}


%\citet{ullrich2021} uses reference counting with 

%borrowed references in the Lean programming language. Their algorithm 

%\citet{reinking2021} 

%Both Lean and Koka are, however, eagerly evaluated. Lazy languages...


\section{Graph Reduction Intermediate Notation}
\begin{figure}[htpb]
\centering
\begin{empheq}[box=\fbox]{align*}
&\begin{array}{l l l l}
term, t & ::=      & term \; ; \; \lambda \rightarrow lalt                     & \; \text{binding} \\
        & \; \mid  & \texttt{case} \; val \; \texttt{of} \; term \; \{calt\}*  & \; \text{case} \\
        & \; \mid  & val \; \{val\}*                                           & \; \text{application} \\
        & \; \mid  & \texttt{unit} \; val                                      & \; \text{return value} \\
        & \; \mid  & \texttt{store} \; val                                     & \; \text{allocate new heap node} \\
        & \; \mid  & \texttt{fetch} \; \{tag\} \; n \;  \{i\}                  & \; \text{load heap node} \\
        & \; \mid  & \texttt{update} \; \{tag\} \; n \; \{i\} \; val           & \; \text{overwrite heap node} \\
        & \; \mid  & \texttt{unreachable}                                      & \; \text{unreachable} \\
\end{array} \end{empheq}
\caption{GRIN syntax. }
\label{fig:spec}
\end{figure}

All syntactically correct expressions are not valid. For example, the value at the function 
position must be either a top-level function or a primitive. It cannot be a variable because
there are no indirect function calls in GRIN. Likewise, top-level functions cannot be passed 
as arguments because GRIN is a first order langauge.

\section{Precise reference counting}

In GRIN, we have to be explicit about allocations, retriving nodes from the heap, and overwriting heap nodes.
As a result, the Perceus primitives \texttt{dup}, \texttt{drop}, \texttt{is-unique}, \texttt{decref}, and \texttt{free} 
can be described with GRIN's exisiting constructs.\footnote{This only possible in our slightly modified version of 
GRIN. \citeauthor{boquist1999}'s original specification of GRIN could not overwrite individual fields of a heap node 
using \texttt{update}.}
Although, GRIN does not have a primitive for freeing memory, this can be simulated by a function call 
to libc's \texttt{free}.
 


\section{Result}

\section{Relevant Work}
% Lean and Koka: reference counting and reuse
% Swift: automatic reference counting (ARC), and plans of borrowing
% AST \& Rust: safe manual memory management through proofs and the borrow checker, respectively

\section{Conclusion and Future Work}
\begin{itemize}
\item Add reuse and borrowing
\item Utilse GRIN's whole program compilation strategy and the heap points-to analysis to statically determine unshared values during compile time, and thus minimizing the number of reference countinging operations.
\item It would also be intresting to utilse 0-modality (erasure) in Agda's type system, and later also 1-modality when Agda gets it.
\item Currently decref and dup are primitives is GRIN however it would be intresting to adjust update
      with offsets, similiar to fetch.
\item In the current naive implementation drop enumerates all the possible tags 
\end{itemize}

% Lambda lifting
% The rest of the GRIN transformations and optimizations
% Partial applications (P-tags and the apply operation)
% Different allocator? mimalloc?
% Using type information (multiplicites: 0, (1), ω)

\printbibliography

\end{document}

